\documentclass{article}

% Language setting
% Replace `english' with e.g. `spanish' to change the document language
\usepackage[english]{babel}

% Set page size and margins
% Replace `letterpaper' with`a4paper' for UK/EU standard size
\usepackage[letterpaper,top=2cm,bottom=2cm,left=3cm,right=3cm,marginparwidth=1.75cm]{geometry}

% Useful packages
\usepackage{amsmath}
\usepackage{graphicx}
\usepackage[colorlinks=true, allcolors=blue]{hyperref}
\usepackage{tikz}
\usetikzlibrary{shapes.geometric, arrows}

\tikzstyle{startstop} = [rectangle, rounded corners, minimum width=3cm, minimum height=1cm, text centered, draw=black, fill=red!30]
\tikzstyle{io} = [trapezium, trapezium left angle = 70, trapezium right angle=110, minimum width=3cm, minimum height=1cm, text centered, draw=black, fill=blue!30]
\tikzstyle{process} = [rectangle, minimum width=3cm, minimum height=1cm, text centered, draw=black, fill=orange!30]
\tikzstyle{decision} = [diamond, minimum width=3cm, minimum height=1cm, text centered, draw=black, fill=green!30]
\tikzstyle{invisible} = [rectangle, minimum width=0cm, minimum height=0cm, text centered, draw=white, fill=white]
\tikzstyle{arrow} = [thick,->,>=stealth]

\title{CPE 2073: Lunar Lander Simulation}
\author{Yasir Choudhury \\ orb234}

\begin{document}
\maketitle

\section{Introduction}

In 1959, Robert Noyce ushered in the Silicon Age with the invention of the monolithic Integrated Circuit, or IC. Through sequential combinations of these ICs, possibilities for in-flight adjustment and complex thrust control beyond human capabilities became feasible. The Apollo Guidance Computer was the first to realize these capabilities and successfully control the Apollo Command Module and Apollo Lunar Module throughout its mission in 1969. Weighing just 70 pounds when its predecessors were weighed in tons, it became clear that scaling down in size was just as much a possibility as scaling up in power.

Since then, computational capability available to the average person has exponentially increased. Vacuum tubes shrunk into silicon chips, and computers could be rapidly configured and re-configured through the emergence of software. What was once an incredibly complex task for a team of NASA scientists can now be roughly modelled by an undergraduate college student. The following report walks through the design of a rudimentary guidance system written in the C programming language.

\section{Problem Statement}

The designed guidance system includes two modes: one with manual pilot input and an autopilot mode to determine the best values for thrust so that the pilot lands safely every time.

The first mode acts as an open loop control system. The user can input any desired thrust value, observe the change in altitude and velocity in accordance with the thrust, and make a judgement on what value to enter in next.

Alternatively, the user can hand over control to a closed-loop automated system. This autopilot mode determines the best thrust value per game tick based on pre-determined output formulas.

Regardless of the control method chosen, the lunar lander must land safely on the surface of the Moon. Both methods must reach a final velocity between -1 and 0 m/s when the value for altitude reaches zero.

\section{Design}

A data flow graph was constructed to understand the overall flow of information throughout the program. This way, the overall operation of the system can be understood without being lost in the details of how everything works. First, the program prints a welcome message and waits for valid input to determine whether the simulated lander will be controlled manually or automatically. Next, the program indefinitely iterates through a while loop that updates values relevant for space flight, prints these values to the screen, and logs them in to a .csv file for later analysis. Once the height value is equal to zero, the program determines if the final velocity is within the bounds for safe landing. Afterward, the program calls a Python script to display the graphed results of the space flight values to the screen.

\begin{center}
\begin{figure}
\begin{tikzpicture}[node distance=2cm]

\node (start) [startstop] {start};
\node (in1) [io, below of=start] {get decision};
\node (dec1) [decision, below of=in1, yshift=-0.5cm] {valid decision?};
\node (dec2) [decision, below of=dec1, yshift=-1.7cm] {man or auto?};
\node (pro1b) [process, right of=dec1, xshift=2.5cm] {print error message};

\node (in2) [io, left of=dec2, xshift=-2.5cm] {get thrust};
\node (pro1c) [process, right of=dec2, xshift=2.5cm] {calculate thrust};

\node (pro1d) [process, below of=in2] {update state};
\node (pro1e) [process, below of=pro1c] {update state};

\node (invis1) [invisible, left of=pro1d, xshift=-1cm] {no};
\node (invis2) [invisible, right of=pro1e, xshift=1cm] {no};


\node (dec3) [decision, below of=pro1d] {h $\leq$ 0?};
\node (dec4) [decision, below of=pro1e] {h $\leq$ 0?};

\node (dec5) [decision, below of=dec2, yshift=-4cm] {-1 $\leq$ v_f $\leq$ 0?};

\node (out1) [io, below of=dec5, xshift=-2.5cm, yshift=-0.5cm] {crash};
\node (out2) [io, below of=dec5, xshift=2.5cm, yshift=-0.5cm] {land};

\node (pro1f) [process, below of=dec5, yshift = -3cm] {export to Python};
\node (stop) [startstop, below of=pro1f] {stop};

\draw [arrow] (start) -- (in1);
\draw [arrow] (in1) -- (dec1);

\draw [arrow] (dec1) -- node[anchor=east] {yes} (dec2);
\draw [arrow] (dec1) -- node[anchor=south] {no}(pro1b);

\draw [arrow] (pro1b) |- (in1);

\draw [arrow] (dec2) -- node[anchor = south] {man} (in2);
\draw [arrow] (dec2) -- node[anchor = south] {auto} (pro1c);

\draw [arrow] (in2) -- (pro1d);
\draw [arrow] (pro1c) -- (pro1e);

\draw [arrow] (pro1d) -- (dec3);
\draw [arrow] (pro1e) -- (dec4);

\draw [arrow] (dec3) -- node[anchor = south west] {yes} (dec5);
\draw [arrow] (dec4) -- node[anchor = south east] {yes} (dec5);

\draw [arrow] (dec5) -| node[anchor=south west] {yes} (out1);
\draw [arrow] (dec5) -| node[anchor=south east] {no} (out2);

\draw [arrow] (out1) |- (pro1f);
\draw [arrow] (out2) |- (pro1f);
\draw [arrow] (pro1f) -- (stop);

\draw [arrow] (dec3) -| (invis1);
\draw [arrow] (invis1) |- (in2);

\draw [arrow] (dec4) -| (invis2);
\draw [arrow] (invis2) |- (pro1c);


\end{tikzpicture}

\caption{Data flow graph}
\end{figure}
\end{center}

\newpage
With the data flow graph constructed and an abstract understanding of the program achieved, further requirements are introduced and refined. 

The user or autopilot mode can choose any amount of thrust, up to a maximum of 45 kN. When the user provides a value for thrust, the input is passed to the \texttt{maxThrust()} function to verify that it does not exceed 45 kN. \texttt{maxThrust()} also verifies that there is enough fuel to service the request for specified thrust. Each kilogram of fuel produces 3000 N of force for one second. For any given thrust value, the fuel burn equation is defined as:

\[Fuel = \frac{T}{3000 m/s}  kg/s\]

where $T$ is thrust in kN. If there is not enough fuel, the function returns a value that corresponds to how much thrust is remaining based on fuel level. From this point onward, no thrust is returned as there is no fuel to generate it.

The autopilot mode automatically calculates thrust through provided control formulas. Thrust output is defined in phases for higher and lower altitudes.

Values relevant to space flight are created and defined in a struct. These values include altitude, velocity, fuel, and mass. After each iteration of the while loop, these values are updated using the \texttt{updateState()} function and the altitude variable is checked to determine how close the lander is to the surface of the moon. If the height value is less than or equal to zero, the velocity is read to determine whether the vehicle crashed or landed safely. Lastly, the Python script is called and the program finishes execution.

\section{Results}


% \subsection{How to add Tables}

% Use the table and tabular environments for basic tables --- see Table~\ref{tab:widgets}, for example. For more information, please see this help article on \href{https://www.overleaf.com/learn/latex/tables}{tables}. 

% \begin{table}
% \centering
% \begin{tabular}{l|r}
% Item & Quantity \\\hline
% Widgets & 42 \\
% Gadgets & 13
% \end{tabular}
% \caption{\label{tab:widgets}An example table.}
% \end{table}

% \subsection{How to add Comments and Track Changes}

% Comments can be added to your project by highlighting some text and clicking ``Add comment'' in the top right of the editor pane. To view existing comments, click on the Review menu in the toolbar above. To reply to a comment, click on the Reply button in the lower right corner of the comment. You can close the Review pane by clicking its name on the toolbar when you're done reviewing for the time being.

% Track changes are available on all our \href{https://www.overleaf.com/user/subscription/plans}{premium plans}, and can be toggled on or off using the option at the top of the Review pane. Track changes allow you to keep track of every change made to the document, along with the person making the change. 

% \subsection{How to add Lists}

% You can make lists with automatic numbering \dots

% \begin{enumerate}
% \item Like this,
% \item and like this.
% \end{enumerate}
% \dots or bullet points \dots
% \begin{itemize}
% \item Like this,
% \item and like this.
% \end{itemize}

% \subsection{How to write Mathematics}

% \LaTeX{} is great at typesetting mathematics. Let $X_1, X_2, \ldots, X_n$ be a sequence of independent and identically distributed random variables with $\text{E}[X_i] = \mu$ and $\text{Var}[X_i] = \sigma^2 < \infty$, and let
% \[S_n = \frac{X_1 + X_2 + \cdots + X_n}{n}
%       = \frac{1}{n}\sum_{i}^{n} X_i\]
% denote their mean. Then as $n$ approaches infinity, the random variables $\sqrt{n}(S_n - \mu)$ converge in distribution to a normal $\mathcal{N}(0, \sigma^2)$.


% \subsection{How to change the margins and paper size}

% Usually the template you're using will have the page margins and paper size set correctly for that use-case. For example, if you're using a journal article template provided by the journal publisher, that template will be formatted according to their requirements. In these cases, it's best not to alter the margins directly.

% If however you're using a more general template, such as this one, and would like to alter the margins, a common way to do so is via the geometry package. You can find the geometry package loaded in the preamble at the top of this example file, and if you'd like to learn more about how to adjust the settings, please visit this help article on \href{https://www.overleaf.com/learn/latex/page_size_and_margins}{page size and margins}.

% \subsection{How to change the document language and spell check settings}

% Overleaf supports many different languages, including multiple different languages within one document. 

% To configure the document language, simply edit the option provided to the babel package in the preamble at the top of this example project. To learn more about the different options, please visit this help article on \href{https://www.overleaf.com/learn/latex/International_language_support}{international language support}.

% To change the spell check language, simply open the Overleaf menu at the top left of the editor window, scroll down to the spell check setting, and adjust accordingly.

% \subsection{How to add Citations and a References List}

% %You can simply upload a \verb|.bib| file containing your BibTeX entries, created with a tool such as JabRef. You can then cite entries from it, like this: \cite{1}. Just remember to specify a bibliography style, as well as the filename of the \verb|.bib|. You can find a \href{https://www.overleaf.com/help/97-how-to-include-a-bibliography-using-bibtex}{video tutorial here} to learn more about BibTeX.

% %If you have an \href{https://www.overleaf.com/user/subscription/plans}{upgraded account}, you can also import your Mendeley or Zotero library directly as a \verb|.bib| file, via the upload menu in the file-tree.

% \subsection{Good luck!}

% We hope you find Overleaf useful, and do take a look at our \href{https://www.overleaf.com/learn}{help library} for more tutorials and user guides! Please also let us know if you have any feedback using the Contact Us link at the bottom of the Overleaf menu --- or use the contact form at \url{https://www.overleaf.com/contact}.


\end{document}
